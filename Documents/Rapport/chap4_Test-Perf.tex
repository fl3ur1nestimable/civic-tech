\vskip 0.60cm
\subsection{Méthode de test}
\vskip 0.25cm
\noindent
Nous avons créer les tests en utilisant la bibliothèque pytest et en respectant la méthode Right BICEP vue en cours. Le Git du projet contient un dossier "tests" que nous avons push qui contient tout ce qui concerne les tests effectués des fonctions et des routes.
\vskip 0.25cm
\subsection{Tests}
\subsubsection{Fonctions}
\vskip 0.25cm
\noindent
Parmi toutes les fonctions que nous utilisons pour notre application web, seules certaines étaient pertinentes à tester car les autres étaient très dépendantes des requêtes effectuées dans le but de les appeler.
\vskip 0.25cm
\noindent
En ce qui concerne les fonctions, tous les tests ont été concluants et nous avons pu en déduire que les fonctions se comportaient comme nous l'attendions.
\vskip 0.25cm
\subsubsection{Routes}
\vskip 0.25cm
\noindent
Pour ce qui est des tests des routes, nous avons cherché à tester le contenu et les codes HTTP des requêtes que nous avons fait sur chaque page du site.
\vskip 0.25cm
\noindent
De la même façon que précedemment, nous avons pu en déduire que les routes établies pour notre application se comportent comme nous le prévoyions.
\vskip 0.25cm
\subsection{Complexité}
\vskip 0.25cm
\noindent
Toutes les fonctions que nous avons conçu sont de complexité constante ou linéaire (nous avons veillé à cela).
\vskip 0.25cm