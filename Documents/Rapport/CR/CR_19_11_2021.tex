\vskip 0.75cm

\begin{center}
\begin{tabular}[]{|l|l|}
     \hline 
     Présent & Absent\\
     \hline
     Cheneviere Thibault & Guillot Thom\\ 
     Hashani Elion &\\ Yebouet Antoine &\\
     \hline
\end{tabular}
\end{center}


\vskip 0,75cm

\noindent
\textbf{Ordre du jour}

\begin{enumerate}
    \item Objectif
    \item Avancement du projet
\end{enumerate}

\vskip 0.5cm

\noindent
\textbf{Objectif}\\
\noindent
La séance d'aujourd'hui a pour but de déterminer les fonctionnalités du projet que l'on souhaite implémenter et obtenir une validation de la part des responsables du module. \\ 

\noindent
\textit{Fonctions principales}
\begin{itemize}
    \item Page de News de la ville (en page d'accueil) avec les news des clubs de la ville (sportif, artistique, culturel, …), nouvelle décision des élus (projets qui ont été acceptés).
    \item Système d’aide à la prise de décision : avant de voter un projet, permettre au citoyen de choisir le budget alloué à ce projet, impliquer les citoyens sur les plans urbains de la ville à travers des sondages (les mineurs ne pourront pas voter et il y a un vote unique par personne) et des discussions. 
    \item Système lors d'élection : présenter les programmes des différents candidats à travers un hémicycle (représentation graphique) qui répertorie tous les candidats. Les candidats écriront directement leur programme pour éviter les informations biaisées.
    \item Système d’entraide entre les riverains : faire une plateforme pour poster ses annonces pour demander de l’aide ou proposer son aide (par exemple aide pour faire du bricolage, de la mécanique, aide sur un problème avec du matériel informatique, …)
    \item Système de création de projet par les citoyens : système de budget participatif (possibilité de financer des projets de riverains) et possibilité de promouvoir un projet avec des pétitions pour les faire connaître et financer par les élues.
    \item Système de signalisation des problèmes : chaque citoyen peut signaler des incivilités ou des problèmes liés à la gestion de la ville (problème de circulation sur certains axes, problème de ramassage des poubelles dans certains quartiers, …)

\end{itemize}

\vskip 0.5cm

\noindent
\textbf{Avancement du projet}

\vskip 0.25cm
\noindent
\textit{Etape 1 bis :} \\
Thibault a commencé l'implémentation du login/authentification des candidats.
\\Elion a débuté la rédaction de la charte de projet, accompagné de la gestion du document.

\vskip 0.25cm

\noindent
\textit{Etat de l'art :}
\\Nous avons trouvé plusieurs exemples de la Civic Tech et ses exemples d'implémentations 

\vskip 1cm
\noindent
\textit{TO DO LIST}
\vskip 0.25cm

\begin{itemize}
\item Pour tous :
\begin{itemize}
    \item Attendre la réponse des responsables du modules pour démarrer l'implémentation
    \item Continuer la recherche sur la Civic Tech
    \item Réaliser la matrice SWOT
\end{itemize}
\item Pour Elion :
\begin{itemize}
    \item Terminer la charte et la gestion du document
\end{itemize}
\item Pour Thibault :
\begin{itemize}
    \item Terminer le système de login
\end{itemize}
\end{itemize}