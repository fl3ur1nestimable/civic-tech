\vskip 0.75cm

\begin{center}
\begin{tabular}[]{|l|l|}
     \hline 
     Présent & Absent\\
     \hline
     Cheneviere Thibault & Yebouet Antoine\\ 
     Guillot Thom &\\ Hashani Elion &\\
     \hline
\end{tabular}
\end{center}


\vskip 0,75cm

\noindent
\textbf{Ordre du jour}

\begin{enumerate}
    \item Objectif
    \item Avancement du projet
    \item Description de l'application
\end{enumerate}

\vskip 0.25cm

\noindent
\textbf{Objectif}\\
\noindent
La séance d'aujourd'hui a pour but de clarifier la description de l’application avec les retours fait sur la première proposition. \\ 

\noindent
\textit{Objectif de l'application :}\\
L’objectif de l’application est de faciliter le vote en donnant un accès facile au programme de chaque candidat et en proposant une première analyse du programme qui pourra être affinée par les citoyens les plus investis. Ensuite la localisation des différents bureaux de vote facilite la démarche de vote pour les personnes les plus récalcitrantes.

\vskip 0.25cm

\noindent
\textbf{Avancement du projet}\\
\noindent
\textit{Etape 1 bis :} \\
Thibault a terminé l'implémentation du login/authentification des candidats.
\\Elion a fini la rédaction de la charte de projet et ainsi que celle de la gestion du document et ajoute les étapes du projet au fur et à mesure. Matrice SWOT réalisé.

\vskip 0.25cm

\noindent
\textit{Etape 2 :}\\
Thom a commencé à réaliser la page d'accueil du site

\vskip 0.25cm

\noindent
\textbf{Description de l'application}
\begin{itemize}
    \item Chaque candidat aura des identifiants pour pouvoir se connecter et entrer son programme en ligne.
    \item Ensuite après le référencement d’un programme, une analyse automatique permet de « noter » suivant plusieurs critères le candidats (critère écologique, sociale, économique). Cette notation permet ensuite d’afficher 3 barres sur le site internet plus ou moins rempli (pourcentage de remplissage).
    \item Chaque citoyen peut ensuite influer sur ces notations (dans la limite d’une variation de $\pm 5\%$) après avoir lu le programme d’un candidat (permet d’avoir un avis plus large sur la perception du candidat).
    \item Système de localisation des différents bureaux de vote pour faciliter le vote de chacun
\end{itemize}

\vskip 1cm
\noindent
\textit{TO DO LIST}
\vskip 0.25cm

\begin{itemize}
    \item Continuer la recherche sur la Civic Tech sur des algorithmes déjà existants
    \item Finir l'étape 2
    \item Entamer l'étape 3
    \item Réaliser le Gantt
\end{itemize}