\vskip 0.60cm
\subsection{Contexte du projet}
\vskip 0.25cm
\noindent
Ce projet à été réalisé dans le cadre du module CS54 (Computer Science 54) de la première année du cycle ingénieur sous statut étudiant de TELECOM Nancy.
\vskip 0.25cm
\noindent
L'objectif de ce projet est d'utiliser nos connaissances des différentes sections du module (Algorithmique, Bases de données et Web) et de mettre en oeuvre les principes de gestion de projet appris dans le cours de gestion de projet pour concevoir une application Web sur le thème de la démocratie participative.
\vskip 0.25cm
\subsection{Organisation du document}
\vskip 0.25cm
\noindent
Dans le chapitre 2, nous ferons une présentation de la démocratie participative et la Civic Tech, et par un état de l'art, nous analyserons les applications existantes de Civic Tech.
\vskip 0.25cm
\noindent
Dans le chapitre 3, nous présentation la conception et l'implémentation de l'application Web, en présentant chacun des trois volets de l'application, bases de données, serveur web et algorithmes de traitement. 
\vskip 0.25cm
\noindent
Dans le chapitre 4, nous présenterons les tests réalisés dans notre application et les performances de cette dernière.
\vskip 0.25cm
\noindent
Dans le chapitre 5 nous présenterons les éléments et outils de gestion de projet que nous avons utilisé, puis dans le chapitre 6, nous ferons un Bilan du projet.
\vskip 0.25cm
\subsection{Principe de fonctionnement général de l'application}
\vskip 0.25cm
\noindent
L'application possède 3 grandes fonctionnalités :
\begin{itemize}
    \item Afficher de façon synthétique les différents candidats à l'élection
    \item Voir l'attention portée par les candidats aux thèmes pertinents pour une élection
    \item Renseigner le chemin le plus rapide vers le bureau de vote le plus proche
\end{itemize}
\vskip 0.25cm
\noindent
Pour ce faire, les candidats seront présentés sous forme de petites cartes constituant un bref résumé du candidat en question en affichant sa description, ses statistiques et le début de son programme. Ils sont ensuite répartis par parti politique afin d'avoir un critère de sélection en plus pour les visiteurs.
\vskip 0.25cm
\noindent
Les statistiques sont établies par analyse du programme du candidat et quantifient l'attention de celui-ci envers les thèmes suivants : Environnement, Social et Économie. Elles reposent sur un nombre de mots-clés présents (ou non) dans le programme.
\vskip 0.25cm
\noindent
De plus, chaque visiteur pourra porter un jugement sur les statistiques de chaque candidat après lecture du programme si le pourcentage déjà présent ne lui convient pas. Après un certain nombre d'avis allant dans le même sens, la statistique du candidat changera et ne pourra plus être modifiée.
\vskip 0.25cm

%Note à moi même : revoir et finir l'intro /!\