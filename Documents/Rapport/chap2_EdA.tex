\subsection{Principe de la Civic Tech}
\vskip 0.25cm
\noindent
La Civic Tech est définie comme étant un outil au service de la démocratie participative. Cette technologie permet l'engagement ainsi que la participation des citoyens dans la décision publique. Avec une mise en relation des citoyens, la Civic Tech leur permet de s'engager auprès de leur gouvernement dans le but de travailler ensemble pour le bien public.
\vskip 0.25cm
\subsection{Analyse des exemples de Civic Tech}

\begin{tabularx}{\textwidth}{X|X|X|X|X}
    \color{blue}Application & \color{blue} Créateurs & \color{blue} But & \color{blue} Fonctionnalités & \color{blue} Lieu/Fréquence d'usage  \\
    \hline
    Consul & Mairie de Madrid avec Pablo Soto, développé en 2016 & Citoyens font remonter des idées aux mairies; campagne de budget participatif & Budget Participatif, Vote, Débat proposé par les utilisateurs, Pétition, Propositions & Vie de tous les jours et dans le monde entier mais principalement en Espagne \\
    \hline
    Assembl & Bluenove & Construire la prochaine constitution européenne avec les citoyens Européens,
    débats et consultations en ligne à grande échelle & Vote, Questionnaire, Débat proposé par les utilisateurs, Prise de décision & Europe et vie quotidienne avec cycle de débat sur 10 semaines.\\
    \hline
    DemocracyOS & Développeurs et politologues en Argentine & Accompagner et rassembler des mouvements citoyens, des institutions, des start-ups, des associations & Consultation publique, Vote, Proposition, Débat proposé par les utilisateurs, Prise de décision & L'application a été reprise dans le monde entier et utilisable à tout moment.\\
    \hline
    Maires\&Citoyens & Créée en 2016 par 2 azuréens & Épauler les maires et élus
    à communiquer efficacement et en temps réel avec les habitants de la commune & Alertes communales, sondages anonymes, suggestions citoyennes, signalements urbains et groupes de discussion & Près de 250 communes en France l'utilise et peut être utilisé quotidiennement\\
    \hline
    DigitaleBox & Fondée en 2013 par Vincent Moncenis & Logiciel de gestion des relations avec les électeurs et de stratégie électoralex & Gestion des réseaux sociaux, communications ciblées, organisation de communautés & En France, surtout utilisé lors des campagnes éléctorales\\
    \hline
    FluiCity & 	Julie de Pimodan \& Nicolas de Briey \& Jonathan Meiss & Facilite la mise en place de nos consultations citoyennes et renforce leur impact & Organiser des conseils, budget participatif, événement, signalement, proposition & Plus de 100 villes \& régions en France et en Belgique, peut être utilisé quotidiennement.\\
\end{tabularx}

\vskip 0.25cm
\subsection{Choix de notre application}
\vskip 0.25cm
\noindent
Suite à plusieurs brainstorming réalisés ainsi que des recherches Web, nous avons pu remarquer que notre société cherche à accéder aux informations de manière instantanée et guidée, avec un effor minimal. On observe une abstention des jeunes plus forte à chaque élection que ce soit pour les départementales, les régionales ou même les présidentielles. Il fallait ainsi que l'on trouve une approche adaptée à cette problématique.
\vskip 0.25cm
\noindent
C'est pour cela qu'une application, s'inspirant de la Civic Tech, permettant d'avoir des statistiques sur chaque candidat répondait à la problématique évoqué précédemment. Cette application est accessible à tous, ludique : avec le design des cartes résumant les candidats, ainsi que le système de vote qui permet de réévaluer les statistiques d'un candidat ; et permet aux utilisateurs de choisir le candidat, pour lequel ils vont voter, plus facilement selon leurs critères.
\vskip 0.25cm