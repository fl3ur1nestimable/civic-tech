\vskip 0.60cm
\subsection{Bilan global du projet}
\vskip 0.25cm
\noindent
\begin{tabularx}{\textwidth}{|X|X|}
    \hline 
    Travail attendu & Travail réalisé \\
    \hline
    Faciliter la démocratie participative locale en prenant en compte les services déjà proposés par la Civic Tech afin de proposer une innovation dans ce domaine. Le but est de s'appuyer sur les outils donnés au cours de l'année : une application web, du code Python et une base de données.
    & 
    Application web qui répertorie chaque candidat inscrits aux élections locales sous forme de cartes contenant des informations avec lesquelles peuvent intéragir de simples visiteurs. Les cartes synthétisent l'attention que porte le candidat sur différents thèmes afin de faciliter le choix de vote. Une intéraction sur ces statistiques est possible afin de mieux correspondre à la vision qu'ont les visiteurs sur le candidat en question.
    \\
    \hline
\end{tabularx}
\vskip 0.60cm
\noindent
\begin{tabularx}{\textwidth}{|X|X|X|}
    \hline
     & Points positifs & Points négatifs \\
    \hline
    \textbf{Écriture du code} & 
    Python est un des langages les plus utilisés de nos jours ce qui facilite grandement la recherche d'informations et de solutions concernant le code. Python est intuitif et très compréhensible de par son système d'indentation.
    \vskip 0.25cm
    HTML, CSS et Bootstrap 5 sont pratiques pour la création de site et sont facilement reliables à Python grâce à des outils comme Flask.
    \vskip 0.25cm
    La base de donnée est facilement accessible grâce à SQLite3 en python ce qui permet l'utilisation directe des informations dans les fonctions Python créées pour le bon fonctionnement du site.
    &
    HTML, CSS et Bootstrap 5 sont très spécifiques quand il s'agit de la mise en page du site en lui-même ce qui rend les codes longs et compliqués à mettre en place.
    \vskip 0.25cm
    La base de donnée possède de nombreuses relations de dépendance et il peut souvent arriver d'oublier de relier certaines informations. Il faut donc souvent y faire attention et cela complique la manipulation de celle-ci.
    \\
    \hline
\end{tabularx}
\vskip 0.60cm
\subsection{Bilan du projet par membre}
\vskip 0.25cm
\subsubsection{Cheneviere Thibault}
\noindent
\begin{tabularx}{\textwidth}{|X|X|}
    \hline
    \textbf{Points positifs} & {Nous avions un groupe avec une bonne ambiance et de bonnes idées ce qui a permis de trouver un sujet intéressant et original.} \\
    \hline
    \textbf{Points négatifs} & {Je pense qu'il y a eu un manque de communication au niveau de la répartition du travail ce qui a impacté la performance du groupe.} \\
    \hline
    \textbf{Expérience personnelle} & {J'ai pu mettre en oeuvre les techniques de python, de web et de base de données vues en cours mais aussi réutiliser des notions que j'avais vu avant ce projet. De plus, c'était ma première expérience de travail en groupe sur un projet informatique et je pense que cela me permettra d'améliorer mon approche et mon travail sur les autres projets.} \\
    \hline
    \textbf{Axes d'amélioration} & {Comme je l'ai dit dans les points négatifs, je pense qu'une meilleure communication ainsi qu'une meilleure répartition du travail permettraient de rendre les prochains projets encore meilleurs.} \\
    \hline
\end{tabularx}
\subsubsection{Guillot Thom}
\noindent
\begin{tabularx}{\textwidth}{|X|X|}
    \hline
    \textbf{Points positifs} & Les idées de chacun étaient pertinentes, à la fois pour implémenter de nouvelles fonctionnalités, mais aussi pour corriger/améliorer une déjà existante. L'ambiance du groupe a permis de travailler dans un climat de bonne entente.\\
    \hline
    \textbf{Points négatifs} & La répartition du travail a été difficile, notamment avec la période de vacances qui nous a séparé et nous n'avions pas les mêmes plages horaires pour travailler.\\
    \hline
    \textbf{Expérience personnelle} & J'ai enrichi ma compréhension en HTML, CSS, Bootstrap 5, JavaScript et Python avec les codes que j'ai pu écrire, mais aussi en regardant le fonctionnement des algorithmes de mes camarades.\\
    \hline
    \textbf{Axes d'amélioration} & Il faudrait à l'avenir améliorer la communication et l'évaluation de la faisabilité et la pertinence d'une idée.\\
    \hline
\end{tabularx}
\subsubsection{Hashani Elion}
\noindent
\begin{tabularx}{\textwidth}{|X|X|}
    \hline
    \textbf{Points positifs} & Le groupe était agréable, nous avons su développer plusieurs idées sur le projet. La bonne ambiance ainsi que les idées pertinentes proposées nous ont permis de bien travailler.  \\
    \hline
    \textbf{Points négatifs} & J'ai trouvé que l'on avait mis trop de temps à se répartir les tâches et donc à vraiment se plonger dans le projet.\\
    \hline
    \textbf{Expérience personnelle} & Ce projet m'a permis de découvrir et surtout de comprendre le fonctionnement d'une application, j'ai pu m'excercer et donc m'améliorer dans les différents langages de programmation.\\
    \hline
    \textbf{Axes d'amélioration} & Renforcer davantage la communication pour que tous les membres soient plus impliqués et surtout pour que le projet avance de manière régulière. \\
    \hline
\end{tabularx}
\subsubsection{Yebouet Antoine}
\noindent
\begin{tabularx}{\textwidth}{|X|X|}
    \hline
    \textbf{Points positifs} & Les idées se sont vite construites dans l'équipe et l'ambiance était bonne. La motivation était présente jusqu'au bout.\\
    \hline
    \textbf{Points négatifs} &  Après la phase initiale du projet, il y a eu un manque de communication en ce qui concerne la répartition des tâches.\\
    \hline
    \textbf{Expérience personnelle} & Ce projet m'a permis de bien cerner les concepts et la technique en ce qui concerne la réalisation d'une application Web, dans les 3 volets concernés, et j'ai amélioré mes connaissances dans les langages utilisés.\\
    \hline
    \textbf{Axes d'amélioration} & Améliorer la communication au sein du groupe et aussi une meilleure prise d'initiative en ce qui me concerne.\\
    \hline
\end{tabularx}
\vskip 0.60cm